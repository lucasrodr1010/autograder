%--------------------------------------------------------------
% Example Live-Coding / Practice Coding Assignment Template
%--------------------------------------------------------------
\documentclass[11pt]{article}
\usepackage[utf8]{inputenc}
\usepackage{geometry}
\usepackage{xcolor}
\usepackage{framed}
\usepackage{float}
\usepackage{parskip}
\usepackage{amsmath}
\usepackage{textcomp} % for \textless and \textgreater
\geometry{margin=1in}

% --- Colour palette (UF-inspired) -------------------------------------------
\definecolor{gatorblue}{HTML}{0021A5}
\definecolor{gatororange}{HTML}{FA4616}
\definecolor{consoleblue}{HTML}{1E3FA3}
\definecolor{lightorange}{RGB}{253,238,222}

% --- Highlight helper --------------------------------------------------------
\newcommand{\learnednote}{\colorbox{yellow}{%
\parbox{\dimexpr\linewidth-2\fboxsep}{\textbf{At the top of your file, include:}
file name, short description, input(s), output(s), your name, and
\emph{at least three things} you learned from this assignment as Python comments.}}}

% --- Test-case frame ---------------------------------------------------------
\newenvironment{testcase}{%
  \def\FrameCommand##1{\colorbox{lightorange}{##1}}%
  \MakeFramed{\advance\hsize -1em \FrameRestore}%
}{\endMakeFramed}

% --- Page header -------------------------------------------------------------
\newcommand{\pageheader}{
  \begin{flushleft}\small
  \textbf{COURSE / SECTION HERE}\hfill\textbf{Example LCA/PCA Template}\par
  \hrulefill
  \end{flushleft}
}

\begin{document}
\pagestyle{empty}
\pageheader

%======================================================================
%  ASSIGNMENT N (Short Title)
%======================================================================
\section*{\textcolor{gatororange}{Assignment Title Goes Here}}
% One-sentence objective/overview:
\textit{Goal: Briefly describe what students will build/learn in this exercise.}

\subsection*{Description}
Provide 2--4 sentences of context. State the main task, constraints, and any real-world tie-in. Keep it concise and actionable.

\subsection*{Inputs}
\begin{itemize}
  \item Describe each input (type/format), e.g., \texttt{filename} (string), numeric value, command, etc.
  \item Note any assumptions (e.g., lowercase, no punctuation) if relevant.
\end{itemize}

\subsection*{Outputs}
\begin{itemize}
  \item Describe expected outputs and formatting (e.g., fixed decimals, labels).
  \item Mention special cases (empty input, invalid values) and how to handle them.
\end{itemize}

\subsection*{\underline{Sample Session / Test Case}}
\begin{testcase}\small
\textcolor{consoleblue}{Prompt:} \texttt{\textless{}\textless{} your prompt text \textgreater{}\textgreater{}}\\
\textcolor{consoleblue}{User:} \texttt{\textless{}\textless{} sample input \textgreater{}\textgreater{}}\\
\textcolor{consoleblue}{Output:} \texttt{\textless{}\textless{} sample output \textgreater{}\textgreater{}}\\
\textcolor{consoleblue}{User:} \texttt{\textless{}\textless{} another input \textgreater{}\textgreater{}}\\
\textcolor{consoleblue}{Output:} \texttt{\textless{}\textless{} corresponding output \textgreater{}\textgreater{}}
\end{testcase}

\subsection*{\underline{SPECIFICATIONS}}
\begin{itemize}
  \item Clear, numbered requirements for correctness.
  \item Edge cases to cover.
  \item Performance or style constraints (if any).
  \item \learnednote
\end{itemize}

%========================  END  =============================================
\end{document}
